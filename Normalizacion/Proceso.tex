\documentclass[11pt]{article}

\usepackage[utf8]{inputenc}
\usepackage[spanish]{babel}
\usepackage{fullpage}
\usepackage[top=2cm, bottom=4.5cm, left=2.5cm, right=2.5cm]{geometry}
\usepackage{amsmath, amssymb}
\usepackage{enumitem}
\usepackage{fancyhdr}
\usepackage{hyperref}
\usepackage{graphicx}
\usepackage{subcaption}
\usepackage{float}
\usepackage{hyperref}


\pagestyle{fancyplain}
\headheight 35pt
\rhead{7 de Diciembre de 2019}
\lhead{Fundamentos de Bases de Datos}
\headsep 1.5em

\begin{document}

\begin{center}
	\LARGE{\textbf{Proyecto Final\\Proceso de Normalización}}
\end{center}

\section*{Procedimiento}
Vamos a normalizar todas las relaciones de nuestro esquema de base de datos de acuerdo a la \textbf{Tercera Forma Normal}. Lo primero que haremos es indicar cada relación en nuestra base de datos con su respectivo conjunto de dependencias funcionales. El proceso con el que se obtuvieron dichas dependencias funcionales se encuentra detallado en el documento de Traducción.\\
NOTA: En el documento los nombres de relación y atributos se escribirán sin tildes, de manera que concuerden con los scripts de SQL.
\begin{itemize}
\item Sucursales(numero\_sucursal, calle, numero, ciudad, estado)
      \begin{itemize}
        \item numero\_sucursal $\rightarrow$calle,numero,ciudad,estado
      \end{itemize}
      En esta relación las únicas dependencias funcionales no triviales tienen del lado izquierdo a la llave de la relación (numero\_sucursal), por lo tanto no hay violaciones a la Tercera Forma Normal.
\item Clientes(correo, nombre, apellido\_paterno, apellido\_materno, puntos, calle, numero, estado, ciudad, telefono, numero\_sucursal) 
      \begin{itemize}
        \item correo$\rightarrow$nombre, apellido\_paterno, apellido\_materno, puntos, calle, numero, estado, ciudad, telefono, numero\_sucursal
      \end{itemize}
      De nuevo se presenta el caso de que todas las dependencias funcionales tienen del lado izquierdo una llave y no hay violación a la Tercera Formal Normal.
\item Empleados(rfc, nombre, apellido\_paterno, apellido\_materno, curp,  tipo\_empleado, tipo\_sangre, fecha\_nacimiento, calle, numero, estado, ciudad, cuenta\_bancaria, numero\_seguro, tipo\_transporte, licencia, numero\_sucursal, salario, bonos, fecha\_contratacion)
      \begin{itemize}
        \item rfc $\rightarrow$curp
        \item curp$\rightarrow$rfc
        \item rfc$\rightarrow$nombre, apellido\_paterno, apellido\_materno, tipo\_empleado, tipo\_sangre, fecha\_nacimiento, calle, numero, estado, ciudad, cuenta\_bancaria, numero\_seguro, tipo\_transporte, licencia, numero\_sucursal, salario, bonos
        \item curp$\rightarrow$nombre, apellido\_paterno, apellido\_materno, tipo\_empleado, tipo\_sangre, fecha\_nacimiento, calle, numero, estado, ciudad, cuenta\_bancaria, numero\_seguro, tipo\_transporte, licencia, numero\_sucursal, salario, bonos
      \end{itemize}
      En este caso hay que hacer notar que tanto curp como rfc son llaves candidatas de esta relación, por lo que de nuevo, al todas las dependencias funcionales tener una de ellas del lado izquierdo, no se presentan violaciones a la 3NF.
\item Vendedores(rfc, nombre, telefono)
      \begin{itemize}
        \item rfc$\rightarrow$nombre,telefono
      \end{itemize}
      El rfc es una llave, por lo que no hay violaciones a la 3NF en esta relación.
\item MateriaPrima(id\_articulo, nombre, tipo)
      \begin{itemize}
        \item id\_articulo$\rightarrow$nombre
        \item nombre$\rightarrow$id\_articulo
        \item id\_articulo$\rightarrow$tipo
        \item nombre$\rightarrow$tipo
      \end{itemize}
      Como tanto id\_articulo como nombre forman una llave para esta relación, todas las dependencias funcionales tienen del lado izquierdo una súper llave, por lo que no hay violaciones a la 3NF.
\item Inventarios(id\_articulo, numero\_sucursal, fecha\_compra, precio\_unitario, cantidad, fecha\_caducidad, rfc\_provedor)
      \begin{itemize}
        \item id\_articulo, numero\_sucursal, fecha\_compra$\rightarrow$ precio\_unitario, cantidad, fecha\_caducidad, rfc\_provedor
      \end{itemize}
      Note que las únicas dependencias funcionales tienen a id\_articulo, numero\_sucursal y fecha\_compra del lado izquierdo, y estos atributos forman una llave, por lo que no hay violaciones a la 3NF
\item Tipo(id\_tipo, nombre)
      \begin{itemize}
        \item id\_tipo$\rightarrow$nombre
        \item nombre$\rightarrow$id\_tipo   
      \end{itemize}
      Note que en esta relación tanto nombre como id\_tipo son llaves candidatas y por lo tanto no hay violación a la 3NF. 
\item Platillos(id\_platillo, id\_tipo, nombre)
      \begin{itemize}
        \item id\_platillo$\rightarrow$id\_tipo, nombre
      \end{itemize}
      Note en esta relación todas las dependencias funcionales tienen una llave del lado izquierdo, por lo que no hay violación a la 3NF. 
\item IngredientesPlatillo(id\_platillo, id\_articulo, cantidad)
      \begin{itemize}
        \item id\_platillo, id\_articulo$\rightarrow$cantidad
      \end{itemize}
      Note en esta relación todas las dependencias funcionales tienen una llave del lado izquierdo, por lo que no hay violación a la 3NF. 
\item Salsas(nombre\_salsa, picor)
      \begin{itemize}
        \item nombre\_salsa$\rightarrow$picor
      \end{itemize}
      De nuevo, no hay violaciones a la 3NF
\item IngredientesSalsa(nombre\_salsa, id\_articulo, cantidad)
      \begin{itemize}
        \item nombre\_salsa, id\_articulo$\rightarrow$cantidad
      \end{itemize}
      No hay violaciones a la 3NF
\item Precios(id\_platillo, fecha, precio)
      \begin{itemize}
        \item id\_platillo, fecha$\rightarrow$precio
      \end{itemize}
      No hay violación a la 3NF. 
\item Recomendaciones(id\_platillo, nombre\_salsa)
      \\En este caso sólo tenemos la dependencia funcional trivial (id\_platillo, nombre\_salsa $\rightarrow$ id\_platillo, nombre\_salsa), por lo que no hay violaciones a la 3NF.
\item PresentacionSalsas(nombre\_salsa, tamanio)
      \\En esta relación se presenta el mismo caso que en la anterior (sólo tenemos nombre\_salsa, tamanio $\rightarrow$ nombre\_salsa, tamanio)
\item PreciosSalsas(nombre\_salsa, tamanio, fecha, precio)
      \begin{itemize}
        \item nombre\_salsa, tamanio, fecha$\rightarrow$precio
      \end{itemize}
      No hay violaciones a la 3NF.
\item Promociones(id\_promocion, tipo\_descuento, dia, tipo\_producto)
      \begin{itemize}
        \item id\_promocion, $\rightarrow$ tipo\_descuento, dia, tipo\_producto
      \end{itemize}
      No hay violaciones a la 3NF
\item Pedidos(numero\_ticket, numero\_sucursal, metodo\_pago, no\_mesa, fecha, correo\_cliente)
      \begin{itemize}
        \item numero\_ticket $\rightarrow$ numero\_sucursal, metodo\_pago, no\_mesa, fecha, correo\_cliente
      \end{itemize}
      No hay violaciones a la 3NF
\item ComponentesPedido(numero\_ticket, id\_platillo, id\_promocion, nombre\_salsa, tamanio)
      \begin{itemize}
        \item numero\_ticket $\twoheadrightarrow$ id\_platillo
        \item numero\_ticket $\twoheadrightarrow$ nombre\_salsa, tamanio
        \item numero\_ticket $\rightarrow$ id\_promocion
      \end{itemize}
      En este caso, para la tercer forma normal, sólo tenemos que fijarnos en la última de las dependencias multivaluadas, pues es la única que es funcional. También hay que notar que es esta relación, toda llave candidata debe necesariamente contener a los atriibutos numero\_ticket, id\_platillo, nombre\_salsa y tamanio, pues no están del lado derecho de ninguna dependencia funcional.\\
      Tomando esto en cuenta, hay que notar que numero\_ticket$\rightarrow$id\_promocion es una violación a la 3NF, y por lo tanto, tenemos que proceder con el proceso de Normalización.\\
      En este caso, $F$ consta de una sola DF, $F = \{numero\_ticket  \rightarrow id\_promocion\}$. Este conjunto ya es mínimo, pues es la única dependencia que podemos usar para el cálculo de cerraduras. \\
      Siguiendo el procedimiento de Normalización, para cada DF en $F$ se crea una tabla cuyos atributos sean únicamente los de $F$:
      \begin{itemize}
      \item PromocionesPedido(numero\_ticket, id\_promocion), $F = \{numero\_ticket, id\_promocion\}$
      \end{itemize}
      Y como no hay una relación con los atributos de una llave candidata, y como 
      \begin{align*}      
      &\{numero\_ticket, id\_platillo, nombre\_salsa, tamanio\}+=\\
      &\{numero\_ticket, id\_platillo, nombre\_salsa, tamanio\, id\_promocion\}
      \end{align*}
      Creamos otra relación con los atributos numero\_ticket, id\_platillo, nombre\_salsa y tamanio:
      \begin{itemize}
      \item X(numero\_ticket, id\_platillo, nombre\_salsa, tamanio), con sólo dependencias funcionales triviales
      \end{itemize}
      Ahora, X no tiene dependencias funcionales no triviales, pero aún sigue teniendo redundancia puesto que estamos repitiendo la información de cada platillo de un pedido un número de veces igual al número de salsas que hubo en el pedido y viceversa. Esto no nos parece ideal, por lo que decidimos, únicamente para esta relación, puesto que no es de interés para la base de datos guardar esta información, Hacer uso de la cuarta forma normal.\\
      Iniciamos con las DMV: numero\_ticket $\twoheadrightarrow$ id\_platillo y numero\_ticket $\twoheadrightarrow$ nombre\_salsa, tamanio. Ambas son violaciones a la 4NF pues son no triviales y del lado izquierdo no tienen una superllave.\\
      Siguiendo el algoritmo de Normalización, elegimos una de ellas. Tomamos numero\_ticket $\twoheadrightarrow$ nombre\_salsa, tamanio. Separamos X en dos relaciones, una que contiene a los atributos de esta DMV, y otra con los atributos restantes y los del lado izqiuerdo de la DMV, que en este caso sólo es numero\_ticket:
      \begin{itemize}
      \item SalsasPedido(numero\_ticket, nombre\_salsa, tamanio), con una única DMV: numero\_ticket $\twoheadrightarrow$ nombre\_salsa, tamanio.
      \item PlatillosPedido(numero\_ticket, id\_platillo), con una única DMV: numero\_ticket $\twoheadrightarrow$ id\_platillo.
      \end{itemize}
      En este caso ya no quedan ni dependencias funcionales ni DMV no triviales (pues las dos que teníamos ahora tienen a todos los atributos de sus respectivas relaciones), y por lo tanto, la cuarta forma normal de X es la anterior. 
      Así, DetallesPedido queda separada en tres relaciones:
      \begin{itemize}
      \item SalsasPedido(numero\_ticket, nombre\_salsa, tamanio)
      \item PlatillosPedido(numero\_ticket, id\_platillo)
      \item PromocionesPedido(numero\_ticket, id\_promocion)
      \end{itemize}
\end{itemize}
\end{document}